\documentclass[11pt]{aghdpl}
% \documentclass[en,11pt]{aghdpl}  % praca w języku angielskim

% Lista wszystkich języków stanowiących języki pozycji bibliograficznych użytych w pracy.
% (Zgodnie z zasadami tworzenia bibliografii każda pozycja powinna zostać utworzona zgodnie z zasadami języka, w którym dana publikacja została napisana.)
\usepackage[english,polish]{babel}

% Użyj polskiego łamania wyrazów (zamiast domyślnego angielskiego).
\usepackage{polski}

\usepackage[utf8]{inputenc}

% dodatkowe pakiety

\usepackage{mathtools}
\usepackage{amsfonts}
\usepackage{amsmath}
\usepackage{amsthm}
\usepackage{algpseudocode}
\usepackage{algorithm}
\usepackage{listings}
\usepackage{color}
\usepackage{tikz}
\usepackage[T1]{fontenc}
%\usepackage[]{algorithm2e}

% --- < bibliografia > ---

\usepackage[
style=numeric,
sorting=none,
%
% Zastosuj styl wpisu bibliograficznego właściwy językowi publikacji.
language=autobib,
autolang=other,
% Zapisuj datę dostępu do strony WWW w formacie RRRR-MM-DD.
urldate=iso8601,
% Nie dodawaj numerów stron, na których występuje cytowanie.
backref=false,
% Podawaj ISBN.
isbn=true,
% Nie podawaj URL-i, o ile nie jest to konieczne.
url=false,
%
% Ustawienia związane z polskimi normami dla bibliografii.
maxbibnames=3,
% Jeżeli używamy BibTeXa:
backend=bibtex
]{biblatex}

\usepackage{csquotes}
% Ponieważ `csquotes` nie posiada polskiego stylu, można skorzystać z mocno zbliżonego stylu chorwackiego.
\DeclareQuoteAlias{croatian}{polish}

\addbibresource{bibliografia.bib}

% Nie wyświetlaj wybranych pól.
%\AtEveryBibitem{\clearfield{note}}


% ------------------------
% --- < listingi > ---

% Użyj czcionki kroju Courier.
\usepackage{courier}

\lstloadlanguages{TeX}

\lstset{
	literate={ą}{{\k{a}}}1
           {ć}{{\'c}}1
           {ę}{{\k{e}}}1
           {ó}{{\'o}}1
           {ń}{{\'n}}1
           {ł}{{\l{}}}1
           {ś}{{\'s}}1
           {ź}{{\'z}}1
           {ż}{{\.z}}1
           {Ą}{{\k{A}}}1
           {Ć}{{\'C}}1
           {Ę}{{\k{E}}}1
           {Ó}{{\'O}}1
           {Ń}{{\'N}}1
           {Ł}{{\L{}}}1
           {Ś}{{\'S}}1
           {Ź}{{\'Z}}1
           {Ż}{{\.Z}}1,
	basicstyle=\footnotesize\ttfamily,
}

% ------------------------

\AtBeginDocument{
	\renewcommand{\tablename}{Tabela}
	\renewcommand{\figurename}{Rys.}
}

% ------------------------
% --- < tabele > ---

\usepackage{array}
\usepackage{tabularx}
\usepackage{multirow}
\usepackage{booktabs}
\usepackage{makecell}
\usepackage[flushleft]{threeparttable}

% defines the X column to use m (\parbox[c]) instead of p (`parbox[t]`)
\newcolumntype{C}[1]{>{\hsize=#1\hsize\centering\arraybackslash}X}


%---------------------------------------------------------------------------

\author{Kamil Bienek}
\shortauthor{K. Bienek}

%\titlePL{Przygotowanie bardzo długiej i pasjonującej pracy dyplomowej w~systemie~\LaTeX}
%\titleEN{Preparation of a very long and fascinating bachelor or master thesis in \LaTeX}

\titlePL{Rozszerzenie kompilatora języka C/C++ do obsługi 8-bitowego procesora RISC}
\titleEN{C/C++ compiler extension for 8-bit RISC processors}


\shorttitlePL{Rozszerzenie kompilatora języka C/C++ do obsługi 8-bitowego procesora RISC} % skrócona wersja tytułu jeśli jest bardzo długi
\shorttitleEN{C/C++ compiler extension for 8-bit RISC processors}

\thesistype{Praca dyplomowa inżynierska}
%\thesistype{Master of Science Thesis}

\supervisor{dr. inż. Jakub Grela}
%\supervisor{Marcin Szpyrka PhD, DSc}

\degreeprogramme{Informatyka}
%\degreeprogramme{Computer Science}

\date{2021}

\department{Katedra Informatyki Stosowanej}
%\department{Department of Applied Computer Science}

\faculty{Wydział Elektrotechniki, Automatyki,\protect\\[-1mm] Informatyki i Inżynierii Biomedycznej}
%\faculty{Faculty of Electrical Engineering, Automatics, Computer Science and Biomedical Engineering}

\acknowledgements{Serdecznie dziękuję \dots tu ciąg dalszych podziękowań np. dla promotora, żony, sąsiada itp.}


\setlength{\cftsecnumwidth}{10mm}

%---------------------------------------------------------------------------
\setcounter{secnumdepth}{4}
\brokenpenalty=10000\relax


\lstdefinestyle{customasm}{
    belowcaptionskip=1\baselineskip,
    frame=single, 
    frameround=tttt,
    xleftmargin=\parindent,
    language=[x86masm]Assembler,
    basicstyle=\footnotesize\ttfamily,
    commentstyle=\itshape\color{green!60!black},
    keywordstyle=\color{blue!80!black},
    identifierstyle=\color{red!80!black},
    tabsize=4,
    numbers=left,
    numbersep=8pt,
    stepnumber=1,
    numberstyle=\tiny\color{black}, 
    columns = fullflexible,
}

\begin{document}

\titlepages

% Ponowne zdefiniowanie stylu `plain`, aby usunąć numer strony z pierwszej strony spisu treści i poszczególnych rozdziałów.
\fancypagestyle{plain}
{
	% Usuń nagłówek i stopkę
	\fancyhf{}
	% Usuń linie.
	\renewcommand{\headrulewidth}{0pt}
	\renewcommand{\footrulewidth}{0pt}
}

\setcounter{tocdepth}{2}
\tableofcontents
\clearpage

%\include{rozdzial3}
%\include{rozdzial1}
%\include{rozdzial2}
%\include{tests}

\chapter{Wstęp}
\section{Cele pracy}
\subsection{Rozszerzenie kompilatora}
\subsection{Wizualizacja}
\section{Zawartość pracy}

\chapter{Analiza literatury?}

\chapter{Procesor}
\section{Architektura RISC}
\section{Rejestry}
\section{Instrukcje}
\section{Format instrukcji}

\chapter{Kompilator}
\section{Struktura}
\section{AVR}
\subsection{AVR-gcc}
\subsection{instrukcje?}
\section{MOS 6502}
\subsection{Instrukcje}

\chapter{Konwerter}
\section{Struktura}
\section{Działanie}

\chapter{Implementacja operacji arytmetycznych}
\section{Operacje na liczbach całkowitych}
\subsection{Dodawanie dwóch liczb ośmiobitowych}
\iffalse
\begin{algorithm}
\caption{An algorithm with caption}\label{alg:cap}
\begin{algorithmic}
\Require $a \geq 0, b \geq 0$
\Ensure $c = a + b$
\State LIL [B]
\State LIH [B]
\State MMA
\State LDA
\State MBA
\State LIL [A]
\State LIH [A]
\State MMA
\State LDA
\State CLC
\State ADL
\State ADH
\State LIL [C]
\State LIH [C]
\State MMA
\State MAC
\State STA

\end{algorithmic}
\end{algorithm}
\fi
\subsection{Odejmowanie dwóch liczb ośmiobitowych}
\iffalse
\begin{algorithm}
\caption{An algorithm with caption}\label{alg:cap}
\begin{algorithmic}
\Require $a \geq 0, b \geq 0$
\Ensure $c = a - b$
\State LIL [B]
\State LIH [B]
\State MMA
\State LDA
\State NOT
\State MAC
\State MBA
\State LIL [A]
\State LIH [A]
\State MMA
\State LDA
\State SEC
\State ADL
\State ADH
\State LIL [C]
\State LIH [C]
\State MMA
\State MAC
\State STA

\end{algorithmic}
\end{algorithm}
\fi
\subsection{Mnożenie dwóch liczb ośmiobitowych}
\iffalse
\begin{algorithm}
\caption{An algorithm with caption}\label{alg:cap}
\begin{algorithmic}
\Require $a \geq 0, b \geq 0$
\Ensure $c = a * b$

\State LIL [B]
\State LIH [B]
\State MMA
\State LDA
\State SHR
\State LIL 0x0
\State LIH 0x0
\State MBA
\State ADL
\State ADH
\State LIX mul
\State MMA
\State JNE
\State LIL ?
\State LIH ?
\State MMA
\State LDA
\State CLC
\State ADL
\State ADH
\State MAC
\State STA
\State 
\State 
\State 
\State 
\State 

\State ...
\State mul:
\State LIL ?
\State LIH ?
\State MMA
\State LDA
\State 
\State 
\State 
\State 
\State 
\State 
\State 
\State 

\end{algorithmic}
\end{algorithm}
\fi
\section{Liczby zmiennopozycyjne w standardzie IEEE-754}

\section{Dodawanie liczb zmiennopozycyjnych}
\subsection{Pseudokod}
\begin{algorithm}

\caption{IEEE-754 addition}
\iffalse
\KwData{ a - normalized float , b - normalized float}
\KwResult {c = a + b}
\fi
\begin{algorithmic}[1]
\Require a - normalized float \\
		 b - normalized float
\Ensure c = a + b
\State $NOP$
\State $NOP$
\State $NOP$
\State $NOP$
\State $NOP$
\State $NOP$
\State $NOP$
\State $NOP$
\State $NOP$
\State $NOP$
\State $NOP$
\algstore{myalg}
\end{algorithmic}
\end{algorithm}

\begin{algorithm}
\begin{algorithmic}[1]
\algrestore{myalg}
\State $NOP$
\State $NOP$
\State $NOP$
\State $NOP$
\State $NOP$
\State $NOP$
\State $NOP$
\State $NOP$
\State $NOP$
\State $NOP$
\State $NOP$


\iffalse
\If {a = 0}{
	c ← b
}

\If {b = 0}{
	c ← a
}


asgn $\leftarrow$ a[31]

aexp ← a[30..23]

aman ← a[22..0]

bsgn ← b[31]

bexp ← b[30..23]

bman ← b[22..0]

\If {aexp = 255}{
	\If {a = NaN}{
		c ← NaN
	}
	\If {a = Inf}{
		\If {bexp = 255}{
			\If {b = NaN}{
				c ← NaN
			}
			\If {b = Inf}{
				\eIf {asgn = bsgn}{
					c ← a
				}{
					c ← NaN
				}
			}
		}
	}
}
\fi
\end{algorithmic}
\end{algorithm}

\cleardoublepage
\subsection{UML}

\includegraphics[scale=0.55]{addition_uml}

\begin{lstlisting}[style=customasm, caption={Dodawanie liczb zmiennopozycyjnych}, label=float_addition]
addf: LIX compare_exp
        MMA
        JMP
cmp:       LIL [B]+1
        LIH [B]+1
MMA
LDA
SHL
LIL 0x0
LIH 0x0
MBA
ADL
ADH
MAC
MBA
LIL [B]
LIH [B]
MMA
LDA
SHL
MAC
CLC
OR
LIL [expB]
LIH [expB]
MMA
MAC
STA

LIL [A]+1
LIH [A]+1
MMA
LDA
SHL
LIL 0x0
LIH 0x0
MBA
ADL
ADH
MAC
MBA
LIL [A]
LIH [A]
MMA
LDA
SHL
MAC
CLC
OR
LIL [expA]
LIH [expA]
MMA
MAC
STA

LIL [expB]
LIH [expB]
MMA
LDA
MBA
MAC

XOR
LIX not_equal
MMA
JNE
LIX equal
MMA
JMP

not_equal:
LIL ST
LIH ST
MMA
MAC
STA
LIL ST-1
LIH ST-1
MMA
LIL 0x0
LIH 0x0
OR
MAC
STA

loop:
LIL ST
LIH ST
MMA
LDA
CLC
SHL
MAC
STA
LIL 0x0
LIH 0x0
MBA
ADL
ADH
LIX check
MMA
JNE
LIL ST-1
LIH ST-1
MMA
LDA
SHL
MAC
STA
LIX loop
MMA
JMP
HLT

check:
LIL ST-1
LIH ST-1
MMA
LDA
SHL
LIL 0x0
LIH 0x0
MBA
ADL
ADH
LIX higherB
MMA
JNE
LIX lowerB
MMA
JMP

higherB:
LIL [expA]
LIH [expA]
MMA
LDA
NOT
MAC
MBA
LIL [expB]
LIH [expB]
MMA
LDA
SEC
ADL
ADH
LIL [exp]
LIH [exp]
MMA
MAC
STA
// przenoszenie
LIL [A]+1
LIH [A]+1
MMA
LDA
MBA
LIL 0xF
LIH 0x7
AND
LIL [manA]
LIH [manA]
MMA
MAC
STA
LIL 0x0
LIH 0x0
MBA
LIL [A]+2
LIH [A]+2
MMA
LDA
OR
LIL [manA]+1
LIH [manA]+1
MMA
MAC
STA
LIL [A]+3
LIH [A]+3
MMA
LDA
OR
LIL [manA]+2
LIH [manA]+2
MMA
MAC
STA

higherBloop:
LIL [manA]
LIH [manA]
MMA
LDA
SHR
MAC
STA
LIL [manA]+1
LIH [manA]+1
MMA
LDA
SHR
MAC
STA
LIL [manA]+2
LIH [manA]+2
MMA
LDA
SHR
MAC
STA

LIL [exp]
LIH [exp]
MMA
LDA
MBA
LIL 0x1
LIH 0x0
NOT
MAC
SEC
ADL
ADH
MAC
STA
LIX higherBloop
MMA
JNE
LIX equals
MMA
JMP

lowerB:
LIL [expB]
LIH [expB]
MMA
LDA
NOT
MAC
MBA
LIL [expA]
LIH [expA]
MMA
LDA
SEC
ADL
ADH
LIL [exp]
LIH [exp]
MMA
MAC
STA
// przenoszenie
LIL [B]+1
LIH [B]+1
MMA
LDA
MBA
LIL 0xF
LIH 0x7
AND
LIL [manB]
LIH [manB]
MMA
MAC
STA
LIL 0x0
LIH 0x0
MBA
LIL [B]+2
LIH [B]+2
MMA
LDA
OR
LIL [manB]+1
LIH [manB]+1
MMA
MAC
STA
LIL [B]+3
LIH [B]+3
MMA
LDA
OR
LIL [manB]+2
LIH [manB]+2
MMA
MAC
STA

lowerBloop:
LIL [manB]
LIH [manB]
MMA
LDA
SHR
MAC
STA
LIL [manB]+1
LIH [manB]+1
MMA
LDA
SHR
MAC
STA
LIL [manB]+2
LIH [manB]+2
MMA
LDA
SHR
MAC
STA

LIL [exp]
LIH [exp]
MMA
LDA
MBA
LIL 0x1
LIH 0x0
NOT
MAC
SEC
ADL
ADH
MAC
STA
LIX lowerBloop
MMA
JNE
LIX equals
MMA
JMP

equal:
//---dodawanie mantys
LIL [B] + 3
LIL [B] + 3
MMA
LDA
MBA
LIL [A] + 3
LIH [A] + 3
MMA
LDA
CLC
ADL
ADH
LIL [C] + 3
LIH [C] + 3
MMA
MAC
STA

LIL [B] + 2
LIL [B] + 2
MMA
LDA
MBA
LIL [A] + 2
LIH [A] + 2
MMA
LDA
ADL
ADH
LIL [C] + 2
LIH [C] + 2
MMA
MAC
STA

LIL [B] + 1
LIL [B] + 1
MMA
LDA
MBA
LIL 0xF
LIH 0x7
AND
LIL [C] + 1
LIH [C] + 1
MMA
MAC
STA
LIL [A] + 1
LIH [A] + 1
MMA
LDA
MBA
LIL 0xF
LIH 0x7
AND
LIL [C] + 1
LIH [C] + 1
MMA
LDA
MBA
MAC
ADL
ADH
LIL 0xF
LIH 0x7
MBA
CLC
MAC
AND
MAC
STA
//---koniec dodawanie mantys

LIL [A] + 1
LIH [A] + 1
MMA
LDA
MBA
LIL 0x0
LIH 0x8
AND
LIL [C] + 1
LIH [C] + 1
MMA
LDA
MBA
MAC
OR
MAC
STA

LIL [A]
LIH [A]
MMA
LDA
MBA
LIL [C]
LIH [C]
MMA
LIL 0x0
LIH 0x0
OR
MAC
STA

HLT


equalall:
//---dodawanie mantys
LIL [manB] + 2
LIL [manB] + 2
MMA
LDA
MBA
LIL [manA] + 2
LIH [manA] + 2
MMA
LDA
CLC
ADL
ADH
LIL [C] + 3
LIH [C] + 3
MMA
MAC
STA

LIL [manB] + 1
LIL [manB] + 1
MMA
LDA
MBA
LIL [manA] + 1
LIH [manA] + 1
MMA
LDA
ADL
ADH
LIL [C] + 2
LIH [C] + 2
MMA
MAC
STA

LIL [manB]
LIL [manB]
MMA
LDA
MBA
LIL [manA]
LIH [manA]
MMA
LDA
ADL
ADH
LIL [C] + 1
LIH [C] + 1
MMA
MAC
STA
//---koniec dodawanie mantys

LIL [B] + 1
LIH [B] + 1
MMA
LDA
MBA
LIL 0x0
LIH 0x8
AND
LIL [C] + 1
LIH [C] + 1
MMA
LDA
MBA
MAC
OR
MAC
STA

LIL [B]
LIH [B]
MMA
LDA
MBA
LIL [C]
LIH [C]
MMA
LIL 0x0
LIH 0x0
OR
MAC
STA

HLT
\end{lstlisting}





\chapter{Wizualizacja}
\section{Symulator}

\chapter{Przykłady}
\section{Przykład A}
\section{Przykład B}
\section{Przykład C}


\chapter{Podsumowanie}

% itd.
% \appendix
% \include{dodatekA}
% \include{dodatekB}
% itd.



\printbibliography

\end{document}
